\documentclass[10pt,a4paper,twocolumn,twoside,titlepage]{article}

\usepackage{ngerman}
\usepackage{graphicx}
\usepackage{blindtext}
\usepackage[subsectionbib]{bibunits}
\usepackage{makeidx}

\usepackage[utf8x]{inputenc}
% always load this as last package:
\usepackage{pmag2}
\usepackage[ngerman]{babel}

%\renewcommand{\forwordname}{Präal}

\author{Karl-Martin Zimmermann}
\title{Satzung der Piratenpartei Sachsen-Anhalt}
\makeindex

\begin{document}
\maketitle

\begin{firstpage}
Dieses Programm stellt die Ziele der Piraten in Hessen dar. Es fußt auf den Prinzipien, aus denen die Piraten entstanden sind: Der Verpflichtung und Verantwortung gegenüber dem Individuum in einer sozialen ethischen Gesellschaft sowie dem Schutz des zur Entfaltung des Menschen immer erforderlichen privaten und unantastbaren Lebensraumes, sowohl in geistiger als auch in materieller Hinsicht.

Gleichzeitig erkennen die Piraten die Regeln eines demokratischen und sozialen Staates als notwendige Grundlage des gesellschaftlichen Lebens. Werden staatliche Regeln erstellt, die ein Abwägen zwischen öffentlichem und privatem Interesse erfordern, sind jedoch die zur Entfaltung der Persönlichkeit notwendigen Räume der privaten Lebensführung jederzeit zu achten.

Kultur und Bildung sind die besten Garanten für eine gefestigte Gesellschaftsstruktur, in der jeder Einzelne sein Tun beurteilen und abwägen kann. Verantwortung und Respekt gegenüber der Gesellschaft und damit auch gegenüber dem Einzelnen sind Grundwerte denen wir uns verpflichtet fühlen.
Die Piraten sind der noch jungen Tradition der digitalen Kommunikation verpflichtet, in deren basisdemokratischen Ausrichtung die Wurzeln der Partei liegen und mit deren selbstordnenden Prinzipien ihre Denkweise in weiten Teilen beschrieben ist.

Aus diesem Hintergrund kommend verfügen die Piraten über sehr hohe Kompetenzen in den von ihnen vertretenen Themen. In anderen landespolitischen Themen, in denen keine ausdrückliche Expertise vorhanden ist, werden die Piraten Vorschläge von Bürgern und anderen Parteien unterstützen, sofern diese mit unseren Grundsätzen vereinbar sind.

Dabei vertreten die Piraten die Haltung, dass der einzelne Abgeordnete der Piraten vor allem seinem Gewissen und im Sinne der in der Präambel genannten Prinzipien dem Schutz der Würde des Menschen verpflichtet ist. Es wird zur Erzielung eines politischen Gewichtes und zur Stabilisierung von Mehrheiten ein Empfehlungsrahmen angestrebt, der aber den einzelnen Abgeordneten nicht von seiner Gewissensentscheidung entbinden kann und darf. Politischem Diskurs ist gegenüber parteipolitischem Druck der Vorrang einzuräumen.
\end{firstpage}

\article{}{Informationelle Selbstbestimmung}
{}
%\index{Jane Doe!Eine \"Uberschrift, lange Version (cc-by-sa)}
\begin{itemize}


\item Jeder Mensch hat eine Privatsphäre, die frei von Überwachung bleiben muss. Ohne begründeten Anfangsverdacht darf es keine Bewegungsprofile, keine staatlichen Übergriffe, keine Lauschangriffe und keine Rasterfahndungen geben.

\item Die Unterwanderungen der Bürgerrechte durch den Staat in Form von Massendatenspeicherung, Rasterfahndungen, Erhebung von biometrischen Daten und Online-Durchsuchungen erfordern ein politisches Gegengewicht. Der Einführung von Überwachungsgesetzen, wie es in Deutschland zur Zeit der Fall ist, treten die Piraten entschieden entgegen. Durch die angedachten und teilweise bereits verabschiedeten Gesetzesvorhaben werden Bürgerrechte aufgehoben; der Rechtsstaat wandelt sich in einen Überwachungsstaat, wie wir es nur von totalitären Regimen kennen.

\item Die Piraten fordern, dass der Staat diese Werkzeuge deinstalliert und unsere Gesellschaft dabei unterstützt, sich weiter zu entwickeln. Die informationelle Selbstbestimmung muss sowohl in der hessischen Verfassung als auch im Grundgesetz der Bundesrepublik Deutschland als Grundrecht verankert werden.

\item Der Staat muss verpflichtet werden, jedem Bürger auf Verlangen über die ihn betreffenden Daten Auskunft zu geben, und bei Überwachung ohne konkreten Verdacht den Bürger von sich aus auf diese Überwachung hinzuweisen. Über die angelegten Datenbanken und die Art und Weise wie mit den Daten umgegangen wird, muss sich die Öffentlichkeit ungehindert informieren können.

\item Die Überprüfbarkeit der gespeicherten Daten soll unverzüglich ohne Terminvorgabe zu den üblichen Öffnungszeiten ermöglicht werden. Die Löschung unberechtigterweise erhobener Daten muss auf Antrag jederzeit möglich sein. Die Löschung muß den betroffenen Bürgern unverzüglich und überprüfbar bestätigt werden.

\item Diese Auskunftspflicht gilt auch für juristische Personen, und muss kostenlos erfolgen.
Daten dürfen nicht unberechtigterweise erhoben werden; sollten sich nachträglich Daten als unberechtigt erhoben herausstellen, sind diese unverzüglich und vollständig zu löschen und die Betroffenen in Kenntnis zu setzen.
\end{itemize}

\section{Abschnitt A: Grundlagen}
\section{Abschnitt B: Finanzordnung}
\section{Abschnitt C: Schiedsgerichtsordnung}
\section{Abschnitt D: Liquid Democracy}

\subsection{Onlinedurchsuchung}

\begin{itemize}
\item Die freie Meinungsäußerung ist ein Eckpfeiler unserer Demokratie und darf nicht durch Onlinedurchsuchungen untergraben werden. Das Missbrauchspotenzial dieser Technik ist so immens, dass die Verantwortbarkeit nicht gegeben ist.
\item Onlinedurchsuchungen stellen zudem eine massive Bedrohung für die Sicherheit der informationellen Infrastruktur dar. Wenn Sicherheitslücken in bestehenden System nicht geschlossen und aufgedeckt oder neue aufgebrochen werden, wird damit die IT-Sicherheitsforschung ad absurdum geführt. Nicht die Begehren der Überwacher, sondern das Interesse des Einzelnen an Informationssicherheit hat Vorrang zu haben.
\end{itemize}

\subsection{Keine Vorratsdatenspeicherung}
\begin{itemize}
\item Die Sicherung des Fernmeldegeheimnisses ist ein wichtiger Grundpfeiler zum Erhalt der Demokratie. Das Fernmeldegeheimnis ist unter anderem durch die Einführung der Vorratsdatenspeicherung bedroht.
\item Die Vorratsdatenspeicherung in der derzeit beschlossenen Form muss ersatzlos gestrichen werden, da diese ein eklatanter Verstoß gegenüber Artikel 10 GG darstellt, in dem systematisch und unter der Annahme einer Kollektivschuld aller an der öffentlich zugänglichen Telekommunikation beteiligter Bürger ausgegangen wird. Auch die Umstände der Telekommunikation sind nach den Leitsätzen im Urteil BvR 1611/96 des Bundesverfassungsgerichtes von Artikel 10 GG geschützt.
\end{itemize}

\subsection{Datensicherheit und Wahrung der Privatsphäre}
\begin{itemize}
\item Zur Gewährleistung von Datensicherheit und Privatsphäre müssen unterstützende Technologien in besonderem Maße gefördert werden. Dazu sollen öffentliche Forschungsprojekte mit dem Ziel gestartet werden, solche Technologien zu entwickeln und einsetzbar zu gestalten. Die Informationsinfrastruktur muss von unabhängigen Fachleuten überprüft werden.
\end{itemize}

\subsection{Keine elektronischen Ausweisdokumente}
\begin{itemize}
\item Biometrische Merkmale dürfen nicht im Zusammenhang mit offiziellen Dokumenten zur Identifikation eines Bürgers gespeichert werden, bereits erhobene Daten müssen gelöscht werden.
\item Die Speicherung von biometrischen Daten im Zusammenhang mit Personaldokumenten bringt in der eigentlichen Sache, der besseren Erkennbarkeit von gesuchten Personen, keinen Sicherheitsgewinn, da dieser von der Zielgruppe dieser Maßnahme umgangen werden kann. Eine erhöhte Fälschungssicherheit, die diese Maßnahmen rechtfertigen würde, ist ebenfalls nicht gegeben. Der einzige Zweck dieser Dokumente ist eine möglichst lückenlose Überwachbarkeit großer Teile der Bevölkerung und ein großer Schritt hin zur systematischen staatlichen Überwachung. Dieser Schritt wird vor allem durch die Möglichkeit der RFID-Technik, Daten berührungslos aus mittlerer Entfernung und ohne Einflußnahme des Trägers abzurufen, ermöglicht.
\end{itemize}

\subsection{Rückbau der Videoüberwachung}
\begin{itemize}
\item Der Ausbau der Videoüberwachung an öffentlichen Plätzen muss gestoppt werden, da er nur scheinbar Sicherheit vermittelt, und eine Verschwendung von Steuergeldern darstellt. Zudem verhindert Videoüberwachung keine Straftaten, in Berlin führte die Einführung der Videoüberwachung und -aufzeichnung auf drei U-Bahn-Linien beispielsweise nicht zu einer sinkenden Kriminalitätsrate, sondern im Gegenteil sogar zu einem leichten Anstieg.
\item Zielführender sind hier eine bessere Ausleuchtung und eine verstärkte persönliche Polizeipräsenz. Videoüberwachung im öffentlichen Raum betrifft nahezu nur unverdächtige Normalbürger. Es gibt keinen Verdacht gegen eine konkrete Person, sondern alle Personen unterstehen einem Generalverdacht.
\end{itemize}

\article{}{Sicherheit}{}

\begin{itemize}
\item Der Bereich Inneres ist für die Piratenpartei von zentraler Bedeutung. Das massive Missbrauchspotenzial durch die installierten Überwachungstechniken und -werkzeuge erfordert allerhöchste Wachsamkeit bei den Beamtinnen und Beamten, die diese Werkzeuge bedienen. Ethische und moralische Grundsätze dürfen nicht übergangen werden. Die Möglichkeit der Nötigung durch Vorgesetzte, gegen geltendes Recht oder gegen ihr eigenes Gewissen zu handeln, muss bei Beamtinnen und Beamten ausgeschlossen sein.
\end{itemize}

\subsection{Innere Sicherheit}
\begin{itemize}
\item Sicherheit entsteht durch Nähe und Vertrauen. Die Piratenpartei Hessen setzt sich für die Aus- und Weiterbildung der Polizei in Hessen ein und will das öffentliche Bild der Polizei als kritisch und bürgernah stärken. Wir brauchen soziale Kompetenz und Persönlichkeit im Polizeidienst. Der Stellenabbau bei der Polizei muss rückgängig gemacht werden, die Finanzierung wird durch freigesetzte Mittel der zu streichenden verfassungswidrigen Projekte Antiterrordatei sowie lebenslang gültige Steuernummer unterstützt.
\end{itemize}

\subsection{Justiz und Polizei}
\begin{itemize}
\item Justitia mag blind sein, aber dies sollte sie wenigstens für jeden Bürger - unabhängig von Wohnort und Vermögen - gleich sein. Kürzungen und Schließungen in der Justiz bedrohen die dritte Säule unseres Rechtsstaates, ein Umstand, den wir als nicht hinnehmbar betrachten.
\item Wir stehen für eine bürgernahe, leistungsfähige und unabhängige Gerichtsversorgung in der Fläche als unverzichtbare Voraussetzung für die Herstellung und den Erhalt des Rechtsfriedens. Wir setzen uns zudem für die Konstanz in den Justizstrukturen und Effizienz im Rahmen der personellen und sachlichen Ausstattung der Gerichte ein.
\item Eine funktionierende und rechtsstaatlich verankerte Polizei als Teil der Exekutive ist unser Ziel. Wir setzen uns für eine Kennzeichnungspflicht für Polizeibeamte ein. Beamte im Einsatz bei Versammlungen sind zu verpflichten, von weitem sicht- und erkennbare Kennzeichen zu tragen. Um dem berechtigten Interesse der Beamten nach Datenschutz Rechnung zu tragen, sind die Kennzeichen pseudonym zu gestalten und dürfen von Einsatz zu Einsatz wechseln. Es muss jederzeit auch im Nachhinein sichergestellt sein, dass mit Hilfe eines richterlichen Beschlusses ein Kennzeichen einer Person zuzuordnen ist.
\end{itemize}

\article{}{Bildung}{}

\begin{itemize}
\item Es ist erforderlich, die in anderen Ländern erfolgreichen Maßnahmen zu erkennen und für Hessen zu adaptieren. Schulen müssen dazu beitragen, soziale Kluften zu überwinden, statt sie zu verstärken. Hierbei müssen Konzepte wie Ganztagsschule, Betreuung durch Pädagogen in freien Nachmittagszeiten aber auch gezielte Förderung leistungsschwacher Schüler diskutiert werden. Möglichkeiten des gemeinsamen Schulweges und die Bevorzugung öffentlicher Transportmittel einerseits im sozialen Kontext, anderseits aber auch im Sinne des Umweltschutzgedankens sind zu prüfen und umzusetzen.
\item Neben einer ausreichenden Zahl von Lehrern ist die Fortbildung der schon vorhandenen Lehrer zu fördern. Gerade Lehrer brauchen Kompetenz im Umgang mit neuen Medien. Es kann nicht sein, dass Kinder, Schüler und Auszubildende diese Aufgabe indirekt wahrnehmen, wie es heute an vielen Bildungseinrichtungen der Fall ist. Hierzu zählt auch die Forderung nach Benutzung von offenen, nach DIN oder ISO zertifizierten Standards zum Datenaustausch sowohl im Unterricht als auch in der Verwaltung.
\item Unterrichtsgarantie Plus darf kein Modell für Unterricht durch unqualifiziertes Lehrpersonal und verschwendete Schulzeit bleiben. Der Festanstellung von Lehrern zur Pufferung krankheits- und fortbildungsbedingter Unterrichtsausfälle ist der Vorzug zu geben. Unterrichtsmaterialien müssen wieder kostenfrei zu Verfügung gestellt werden. Für Ganztagsschulen oder bei Nachmittagsunterricht ist es erforderlich, eine kostenlose Verpflegung mit vollwertigen Mahlzeiten anzubieten. Dass diese auch eher ökologischen Maßstäben entsprechen und nicht von minderwertiger Kantinenqualität sein sollte, ist selbstverständlich. Die Umsetzbarkeit dieser Forderungen zeigen zahlreiche Modelle.
\item Die Einführung der Studiengebühren zur Erhöhung der Effizienz in Hochschulen und Universitäten verfehlte klar ihr Ziel, Mit ihrer Rücknahme ist ein erster Schritt getan. Jetzt sind Finanzierungsmodelle umzusetzen, die eine Benachteiligung sozial und finanziell schwächer gestellter Studenten ausschließen. Als Finanzierungsmodelle für Hochschulen sind neben der öffentlichen Hand auch private und gewerbliche Stiftungen anzustreben. Die freie und uneingeschränkte Zugänglichkeit zu Wissen und Werken, die in Hochschulen erarbeitet wurden, muss für alle öffentlichen Einrichtungen gewährleistet sein. Wirtschaftlich verwertbare Forschungsergebnisse und daraus erzielte Erlöse sollen in den Ausbau von Bildungseinrichtungen fließen.
\end{itemize}

\subsection{Neue Bildungskonzepte}
\begin{itemize}
\item Veränderungen sind durch Streichung oder rot markiert* Die deutsche Bildungspolitik ist in Lagern gefangen, die sich kaum bewegen können. Die hessischen Piraten sehen Bildung als lebenslangen Prozess, der nicht nur auf junge Menschen beschränkt werden sollte. Wir wollen deshalb Rahmenbedingungen einführen, die das immerwährende Lernen in allen Lebensbereichen ermöglichen.
\item Zudem soll endlich wieder der Lernende im Mittelpunkt stehen und zum Hauptakteuer des eigenen Bildungsprozesses werden. Durch mehr Mitbestimmung und Transparenz in den Lehrplänen und Angeboten wollen wir Bildungseinrichtungen als demokratischen und öffentlichen Handlungsraum der Teilnehmer gestalten.
\end{itemize}

\subsection{Menschenbild}
\begin{itemize}
\item Gesellschaftliche Grundlage und Ziel pädagogischer Arbeit ist der mündige Bürger, der in der Lage ist sich seines eigenen Verstandes zu bedienen.
\end{itemize}

\subsection{Zweck von Bildung}
\begin{itemize}
\item Bildung dient - entsprechend den Ideen eines humanistischen Menschenbildes - der Entwicklung der Gesamtpersönlichkeit eines jeden Menschen. Wir gehen davon aus, dass alle Menschen von Natur aus nach Wissen und Erkenntnis streben. Folglich hat Bildung die Prozesse der Selbstbildung und Aufklärung voranzutreiben. Auf der Basis von Interesse und Neugier soll Bildung Reflexionsfähigkeit und eigenes Urteilsvermögen im Sinne der Mündigkeit und einer kritischen Abwägung von Einsichten, Argumenten, dem Überprüfen von Hypothesen und langfristigen Folgen dienen.
Bildung mündet in bewusster und souveräner Handlungsfähigkeit.
\end{itemize}

\subsection{Rolle des Staates}
\begin{itemize}
\item Die Privatisierung staatlicher Bildungseinrichtungen verfehlt das Ziel gesellschaftlich wertvoller Bildung und wird von den Piraten Hessen abgelehnt. Am bestehenden Modell von Schulen in freier Trägerschaft soll nichts geändert werden - hier ist durch die staatliche Aufsicht der bildungspolitischen Verantwortung genüge getan. Eine Einflussnahme von Lobbyisten findet nicht statt.
\item Jeder Mensch muss unabhängig von sozialer und kultureller Herkunft, finanzieller Lage und sonderpädagogischem Förderbedarf die von ihm bevorzugte Bildungsform frei wählen können. Pauschale Ausschlusskriterien sind grundsätzlich abzulehnen.
Bildung wird vom Staat bezahlt und ist nicht auf Drittmittel bzw. Finanzierung der Wirtschaft angewiesen. Alle Bildungseinrichtungen unterliegen staatlicher Kontrolle und sind allen Lernern kostenfrei zugängig.
\end{itemize}

\subsection{Frühkindliche Sozialisation}
\begin{itemize}
\item Kinder und Jugendliche erschließen sich die Welt durch Neugierde und benötigen dafür eine altersgerechte Beziehung in Betreuungs-, Erziehungs- und Bildungsverhältnissen. Sie werden dafür in einer geeigneten Umgebung gefördert.
\item Besondere Bedeutung kommt im Entwicklungsprozess dem Spiel zu, da das Spiel Grundlage allen selbstmotivierten Lernens und eines gesunden Selbstwertgefühls ist.
Kinder brauchen in dieser Zeit vor allem sinnliche Erfahrungs- und Bewegungswelten, Zeit für unmittelbare Erlebnisse, aufrichtige Zuwendung und ein Gefühl der Zugehörigkeit, um sich gesund entwickeln zu können.
\item Erziehungsmaßnahmen und –stile orientieren sich an den Bedürfnissen des Kindes und fördern das Kind im Sinne einer ganzheitlichen Entwicklung: Alle Entwicklungsbereiche (geistige, körperliche, emotional-sinnliche, ästhetisch-kulturelle etc.) werden gleichermaßen altersgerecht und situationsspezifisch gefördert. Dies schließt soziales Leben in Gruppen und die Integration von Kindern in besonderen Lebenslagen und aus verschiedenen Lebenswelten ausdrücklich mit ein.
\item Das pädagogische Personal ist Betreuer, Erzieher und Vorbild zugleich und wird für diese Aufgaben und Rollen qualitativ hochwertig ausgebildet. Die Piraten lehnen die geplante Verkürzung der Erzieher-Ausbildung ab. Pädagogisches Personal wird angemessen bezahlt und erhält Beratung und Supervision als Unterstützung ihrer Arbeit.
\item Eine Vielfalt von Lebensstilen, Einrichtungsformen und pädagogischen Konzepten ist erwünscht, da sie zur Auseinandersetzung anregen und Kindern wie Eltern unterschiedliche Handlungsmöglichkeiten eröffnen. Um eine konstruktive Erziehungsunterstützung zu gewährleisten, arbeiten alle Beteiligten eng zusammen.
Hierfür schafft der Staat die entsprechenden Rahmenbedingungen und sorgt für kostenfreie Zugänge.
\item Jedes Kind hat mit Abschluss des Mutterschutzes das Recht auf kostenlose qualifizierte frühkindliche Betreuung. Das Land ist verpflichtet, dafür zu sorgen, dass für jedes Kind einen Betreuungsplatz in direktem Wohnumfeld zur Verfügung steht.
Ab dem dritten Lebensjahr des Kindes ist den Eltern ein Kindergartenplatz aktiv anzubieten. Darüber hinaus schafft das Land Anreize, die den Kindergartenbesuch für Eltern und Kinder attraktiv machen und wirbt für seine Vorteile. Negative Anreize wie ein Erziehungsgeld werden abgelehnt.
\end{itemize}

\subsection{Schulbildung}
\begin{itemize}
\item edwedes Bildungsangebot – von Krippen bis Schule ist gebührenfrei. Es gibt keine versteckten Kosten für Lernmittel, Bücher, Computer, Kopierkosten, Klassenfahrten, Mittagessen, Förderung u.Ä..

\item Bildung geht vor Copyright bzw. Urheberrecht. Im Rahmen des Unterrichts ist die Kopierbarkeit und der Einsatz jedweden Materials prinzipiell kostenfrei und rechtlich zu gewährleisten.

\item Niemand darf aufgrund von Herkunft, Geschlecht, Wohnort, Einkommen, Alter, Aussehen, Behinderung oder vorhandenen Budgets von einer Schule ausgeschlossen werden. Übersteigt die Nachfrage nach einer bestimmten Schule das Angebot an Plätzen, legt das Schulamt ein Auswahlverfahren fest, das sicherstellt, dass die Vergabe nach objektiven und pädagogischen Kriterien und ohne Berücksichtigung der Person erfolgt.

\item Werden Bildungseinrichtung in freier Trägerschaft, analog den öffentlichen Schulen, durch das Land Hessen gefördert, so dürfen sie über diese Förderung hinaus keine Gebühren oder Schulgeld verlangen.

\item Die Schulen, staatliche wie solche in freier Trägerschaft, erhalten die staatliche Zuweisungen an Lehrerstellen bzw. Finanzmitteln auf der Basis einer Lerngruppenstärke von 20 Schülern

\item Die Regelgrundschulzeit beträgt 6 Jahre.

\item Schulen sind als angebotsorientierte Ganztagsschulen von 07:00 bis 17:00 Uhr zu organisieren (ausgenommen berufliche Schulen). Kernarbeits- und Unterrichtszeit ist von 09:00 bis 15:00 Uhr. Davor und danach sind geeignete Förder- und Forderangebote (Talentförderung, AG's, usw.) anzubieten.

\item Die Piraten fordern die sofortige Umsetzung der in der Behindertenrechtskonvention (BRK), die seit dem 26. März 2009 in Kraft gesetzt wurde. Alle beeinträchtigten Schüler und Schüler mit Behinderung haben das Recht auf den Besuch einer Regelschule (Inklusion). Alle Regelschulen sind somit Inklusionsschulen. Jede Klasse muss bis zu 3 beeinträchtigte Schüler aufnehmen. Jede Inklusionsklasse wird durch einen Förderlehrer unterstützt (durchgängig).

\item An jeder Schule sind pro 100 Schüler
\begin{itemize}
\item Schul-Sozialpädagogen
\item Schul-Psychologen,
\item Speziallehrkräfte, die nach normaler Ausbildung und längerer Schulpraxis eine intensive Ausbildung zu psychologischen Grundlagen, diagnostischer Kompetenz und einer differenzierten Methodenpalette des Förderns absolviert haben
\item sowie Fachkräfte mit einer Grundausbildung als Krankenpfleger / Krankenschwester mit Zusatzausbildung für vorbeugende Gesundheitsarbeit
\end{itemize}

mit mindestens je einem Tag pro Woche einzusetzen.

\item Lernende mit Lernproblemen haben Anspruch auf umfangreiche Förderung durch Speziallehrkräfte.

\item Die Piraten streben die Überwindung des veralteten mehr-gliedrigen Schulsystems an. Im Sinne eines gemeinsamen, sozialen Lernens schließt sich der 6-jährigen Grundschule eine 4-jährige Gemeinschaftsschule an. Hier wird zunächst im gemischten Klassenverband, mit fortschreitendem Alter zunehmend in Interessens-spezifischen sowie Förderungs-orientierten Kursen.
Das schließt die Anwendung unterschiedlicher pädagogischer Konzepte innerhalb einer Schule ein.

\item Schule ist überwachungsfreier Raum. Die Piraten lehnen jede Form von Überwachung von Schülern, egal ob durch Schultrojaner, überwachte Internet-Anbindungen oder in audio-visueller Form ab.
\end{itemize}

\subsection{Beschäftigungsverhältnisse}
\begin{itemize}
\item Die Ausbildung von Lehrkräften und Erzieherinnen/Erziehern ist qualitativ hochwertig und die Arbeit wird entsprechend dotiert.
\item Diee dreijährige Fachschulausbildung für Erzieherinnen/Erzieher wird beibehalten.
\end{itemize}

\subsection{Aus- und Weiterbildung der Lehrkräfte}
\begin{itemize}
\item Die Ausbildung der Lehrkräfte in Hessen bedarf dringend einer Veränderung. Ziele der Piraten sind hier:

\begin{itemize}
\item Potentielle Lehrkräfte sollen viel früher feststellen können, ob sie persönlich für den Unterricht von Kindern und Jugendlichen geeignet sind.

\item Angehende Lehrkräfte sollen noch besser pädagogisch und didaktisch auf den Unterricht vorbereitet werden

I\item n der Praxis stehende Lehrkräfte sollen mit geeigneten Anreizen zur regelmäßigen Weiterbildung und Tätigkeit an verschiedenen Schulen motiviert werden

\item Allen Lehrkräften sollen ihre Stärken und Schwächen aufgezeigt sowie Weiterbildungs-und Entwicklungspotentiale erschlossen werden.

\item Das Verfallen in unreflektierte Verhaltens- und Reaktionsmuster sowie das Entstehen von Stress-Krankheiten und Burn Out soll vermieden werden.
\end{itemize}


Weiterbildungen, Schul- und Fachwechsel und Auszeiten sollen erleichtert und positiv unterstützt werden.
\end{itemize}

\

\printindex
%\listoffigures

\end{document}