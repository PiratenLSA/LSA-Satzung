\section{Abschnitt A: Grundlagen}

\subsection{§ 1 - Name, Sitz und Tätigkeitsgebiet}
\begin{enumerate}
\item Der Landesverband Sachsen-Anhalt der Piratenpartei Deutschland ist ein
untergeordneter Gebietsverband auf Landesebene gemäß der Satzung der
Piratenpartei Deutschland (Bundessatzung) und richtet sich nach den Vorgaben aus
der Satzung der Piratenpartei Deutschland. Er vereinigt Piraten ohne Unterschied
der Staatsangehörigkeit, des Standes, der Herkunft, der ethnischen
Zugehörigkeit, des Geschlechts, der sexuellen Orientierung und des
Bekenntnisses, die beim Aufbau und Ausbau eines demokratischen Rechtsstaates
und einer modernen freiheitlichen Gesellschaftsordnung geprägt vom Geiste
sozialer Gerechtigkeit mitwirken wollen. Totalitäre, diktatorische und
faschistische Bestrebungen jeder Art lehnt die Piratenpartei Deutschland
entschieden ab.

\item Der Landesverband Sachsen Anhalt der Piratenpartei Deutschland führt einen
Namen und eine Kurzbezeichnung. Der Name lautet: Piratenpartei Deutschland
Landesverband Sachsen-Anhalt. Die offizielle Abkürzung des Landesverbandes
Sachsen-Anhalt der Piratenpartei Deutschland lautet: PIRATEN.

\item Der Sitz des Landesverbandes ist Halle. Untergeordnete Gliederungen des
Landesverbandes Sachsen-Anhalt der Piratenpartei Deutschland führen den Namen
Piratenpartei Deutschland verbunden mit ihrer Organisationsstellung und dem
Namen der Gliederung.

\item Das Tätigkeitsgebiet des Landesverbandes Sachsen-Anhalt der Piratenpartei
Deutschland ist das Bundesland Sachsen-Anhalt.

\item Die im Landesverband Sachsen-Anhalt der Piratenpartei Deutschland
organisierten Mitglieder werden geschlechtsneutral als Piraten bezeichnet.
\end{enumerate}

\subsection{§ 2 - Mitgliedschaft}
\begin{enumerate}
\item Mitglied des Landesverbandes ist jedes Mitglied der Piratenpartei Deutschland 
mit angezeigten Wohnsitz in Sachsen-Anhalt.

\item Der Landesverband und jede untere Gliederung führt ein Piratenverzeichnis
auf entsprechender Ebene.
\end{enumerate}

\subsection{§ 3 - Erwerb der Mitgliedschaft}
\begin{enumerate}
\item Der Erwerb der Mitgliedschaft der Piratenpartei Deutschland wird durch die
Bundessatzung geregelt.

\item Jegliche Änderung am Bestand der Mitgliedsdaten muss allen übergeordneten
Gliederungen mitgeteilt werden.
\end{enumerate}

\subsection{§ 4 - Rechte und Pflichten der Piraten}
\begin{enumerate}
\item Um eine Gleichbehandlung aller Piraten im Landesverband zu gewährleisten,
werden die Rechte und Pflichten der Piraten des Landesverbandes allein durch die
Bundessatzung geregelt. Eine hiervon abweichende Regelung durch niedere
Gliederungen ist unzulässig.
\end{enumerate}

\subsection{§ 5 - Beendigung der Mitgliedschaft}
\begin{enumerate}
\item Die Beendigung der Mitgliedschaft ist der niedrigsten Gliederung anzuzeigen.

\item Die Beendigung der Mitgliedschaft in der Piratenpartei Deutschland wird
durch die Bundessatzung geregelt.

\item Die Beendigung der Mitgliedschaft im Landesverband erfolgt durch Wechsel
des Wohnsitzes in ein anderes Bundesland oder durch Beendigung der
Mitgliedschaft in der Piratenpartei Deutschland.
\end{enumerate}

\subsection{§ 6 - Ordnungsmaßnahmen}
\begin{enumerate}
\item Verstößt ein Pirat des Landesverbandes gegen die Satzung oder gegen
Grundsätze oder Ordnung der Piratenpartei Deutschland und fügt ihr damit Schaden
zu, kann der Landesvorstand Ordnungsmaßnahmen verhängen.

\item Vorstand und Schiedsgericht des Landesverbandes behandeln
Ordnungsmaßnahmen gemäß Bundessatzung und Bundes-Schiedsgerichtsordnung.
\end{enumerate}

\subsection{§ 7 - Gliederung}
\begin{enumerate}
\item Der Landesverband Sachsen-Anhalt kann sich in Orts-, Kreis- und
Regionalverbände gliedern.

\item Regionalverbände sind Kreisverbände im Sinne der Bundessatzung, deren
Gebiet sich über mehr als einen politischen Kreis erstreckt. Eine Koexistenz von
Kreis- und Regionalverband auf dem selben Gebiet ist nicht zulässig.

\item Der Gründung eines Kreisverbandes müssen mindestens drei akkreditierte
Piraten aus jedem politischen Kreis mehrheitlich zustimmen. Insgesamt müssen der
Gründung mindestens zehn akkreditierte Piraten mehrheitlich zustimmen.

\item Sofern der zuständige Kreisverband keine anderen Regelungen getroffen hat,
gilt für die Gründung von Ortsverbänden Absatz (3) Satz 2.

\item Gründet sich eine Untergliederung oder ändert ihre Satzung, so muss dem
Landesvorstand die aktuelle Satzung vorgelegt werden.

\item Die Geschäftsordnung des Vorstandes einer Untergliederung ist von allen
Vorstandsmitgliedern zu unterschreiben und dem Landesvorstand in Kopie
vorzulegen. Die Geschäftsordnung ist an geeigneter Stelle online zu stellen.
Änderungen an der Geschäftsordnung sind dem Landesvorstand unverzüglich zu
melden sowie in der Onlineversion zu aktualisieren.

\item Liegt die Kopie der Geschäftsordnung des Vorstandes einer Untergliederung
nicht 4 Wochen nach Wahl gemäß Absatz (6) beim Landesvorstand vor, so gilt der
Vorstand der Untergliederung dauerhaft als beschlussunfähig. In diesem Fall hat
der Landesvorstand unverzüglich eine neue Mitgliederversammlung einzuberufen,
bei der sich nur mit der Neuwahl des Vorstands oder Auflösung des Verbands
befasst werden darf. Bis die Neuwahl des Vorstandes zustande kommt, führt der
Landesvorstand oder vom Landesvorstand beauftragte Personen kommissarisch die
Geschäfte der Untergliederung.
\end{enumerate}

\subsection{§ 8 - Bundespartei und Landesverbände}
\begin{enumerate}
\item Der Landesverband verpflichtet sich, den Regelungen des Bundessatzung
bzgl. des Verhältnisses von Bundespartei und Landesverbänden Folge zu leisten
und seine Untergliederungen zu ebensolchem Verhalten anzuhalten.
\end{enumerate}

\subsection{§ 9 - Organe des Landesverbands}
\begin{enumerate}
\item Organe sind der Vorstand, der Landesparteitag, das Landesschiedsgericht,
die Gebietsversammlung, die Aufstellungsversammlung und die
Gründungsversammlung.

\item Die Gründungsversammlung tagt nur einmal, und zwar am 27.6.2009. 
\end{enumerate}

\subsubsection{§ 9a - Der Vorstand}
\begin{enumerate}
\item Dem Vorstand gehören mindestens vier Piraten an: Der Vorsitzende, der
stellvertretende Vorsitzende, der Schatzmeister und der Generalsekretär. Der
Landesparteitag kann zusätzlich bis zu fünf Beisitzer zu Vorstandsmitgliedern
wählen.

\item Der Vorstand vertritt den Landesverband nach innen und außen. Er führt die
Geschäfte auf Grundlage der Beschlüsse der Parteiorgane.

\item Die Mitglieder des Vorstandes werden vom Landesparteitag oder der
Gründungsversammlung in geheimer Wahl bis zum nächsten ordentlichen
Landesparteitag gewählt oder bis ein neuer Landesvorstand durch einen
ausserordentlichen Parteitag gewählt wird.

\item Der Vorstand tritt in seiner regulären Amtsperiode mindestens sechsmal
zusammen. Er wird vom Vorsitzenden oder bei dessen Verhinderung von einem seiner
Stellvertreter schriftlich (Brief, Email oder Fax) mit einer Frist, die in der
Geschäftsordnung des Vorstandes festgelegt wird und eine Woche nicht
unterschreitet, unter Angabe der Tagesordnung und des Tagungsortes einberufen.
Tagesordnungspunkte, die vor der Frist bekannt sind, werden in die vorläufige
Tagesordnung aufgenommen. Bei außerordentlichen Anlässen kann die Einberufung
auch kurzfristiger erfolgen.

\item Auf Antrag eines Zehntels der Piraten des Landesverbandes Sachsen-Anhalt
kann der Vorstand zum Zusammentritt aufgefordert und mit aktuellen
Fragestellungen befasst werden.

\item Der Vorstand beschließt über alle organisatorischen und politischen Fragen
im Sinne der Beschlüsse des Landesparteitages bzw. der Gründungsversammlung.

\item Der Vorstand gibt sich eine Geschäftsordnung und veröffentlicht diese
angemessen. Sie umfasst u.a. Regelungen zu:
\begin{enumerate}
\item Verwaltung der Mitgliedsdaten und deren Zugriff und Sicherung

\item Aufgaben und Kompetenzen der Vorstandsmitglieder

\item Dokumentation der Sitzungen

\item virtuellen oder fernmündlichen Vorstandssitzungen

\item Form und Umfang des Tätigkeitsberichts

\item Beurkundung von Beschlüssen des Vorstandes
\end{enumerate}

\item Die Führung der Landesgeschäftsstelle wird durch den Vorstand beauftragt
und beaufsichtigt.

\item Der Vorstand liefert zum Landesparteitag einen schriftlichen
Tätigkeitsbericht ab. Dieser umfasst alle Tätigkeitsgebiete der
Vorstandsmitglieder, wobei diese in Eigenverantwortung des Einzelnen erstellt
werden. Wird der Vorstand insgesamt oder ein Vorstandsmitglied nicht entlastet,
so kann der Landesparteitag oder der neue Vorstand gegen ihn Ansprüche gelten
machen. Tritt ein Vorstandsmitglied zurück, hat dieser unverzüglich einen
Tätigkeitsbericht zu erstellen und dem Vorstand zuzuleiten.

\item Tritt ein Vorstandsmitglied zurück bzw. kann dieses seinen Aufgaben nicht
mehr nachkommen, so gehen seine Kompetenzen und Aufgaben, wenn möglich, auf ein
anderes Vorstandsmitglied über.
Der Vorstand gilt als nicht handlungsfähig, wenn 1. mehr als ein Drittel der
Vorstandsmitglieder zurückgetreten sind oder ihren Aufgaben nicht mehr
nachkommen können oder 2. wenn die Aufgaben des Vorsitzenden oder des
Schatzmeisters nicht mehr erfüllt werden können oder 3. der Vorstand sich selbst
für handlungsunfähig erklärt.
In einem solchen Fall ist unverzüglich eine außerordentliche
Mitgliederversammlung einzuberufen und vom restlichen Vorstand zur Weiterführung
der Geschäfte eine kommissarische Vertretung zu ernennen. Diese endet mit der
Neuwahl des gesamten Vorstandes.

\item Tritt der gesamte Vorstand geschlossen zurück oder kann seinen Aufgaben
nicht mehr nachkommen, so fährt der dienstälteste Vorstand der nächst niederen
Gliederung kommissarisch die Geschäfte bis ein von ihm einberufener
außerordentlicher Parteitag schnellstmöglich stattgefunden und einen neuen
Vorstand gewählt hat.
\end{enumerate}

\subsubsection{§ 9b - Der Landesparteitag}
\begin{enumerate}
\item Der Landesparteitag ist die Mitgliederversammlung auf Landesebene.

\item Der Landesparteitag tagt mindestens einmal jährlich. Die Einberufung
erfolgt aufgrund eines Vorstandsbeschlusses. Wenn ein Zehntel der Piraten,
mindestens aber zehn Piraten es beim Vorstand beantragen, muss dieser binnen 2
Wochen einen Parteitag einberufen. Der Vorstand lädt jedes Mitglied schriftlich
(Brief, Email oder Fax) mindestens 4 Wochen vorher ein. Die Einladung hat
Angaben zum Tagungsort, Tagungsbeginn, vorläufiger Tagesordnung und der Angabe,
wo weitere, aktuelle Veröffentlichungen gemacht werden, zu enthalten. Spätestens
1 Wochen vor dem Parteitag sind die Tagesordnung in aktueller Fassung, die
geplante Tagungsdauer und alle bis dahin dem Vorstand eingereichten Anträge im
Wortlaut zu veröffentlichen.

\item Ist der Vorstand handlungsunfähig, kann ein außerordentlicher
Landesparteitag einberufen werden. Dies geschieht schriftlich mit einer Frist
von zwei Wochen unter Angabe der Tagesordnung und des Tagungsortes. Er dient
ausschließlich der Wahl eines neues Vorstandes.

\item Der Landesparteitag nimmt den Tätigkeitsbericht des Vorstandes entgegen
und entscheidet daraufhin über seine Entlastung.

\item Über den Landesparteitag, die Beschlüsse und Wahlen wird ein
Ergebnisprotokoll gefertigt, das von der Protokollführung, der
Versammlungsleitung und dem neu gewählten Vorsitzenden oder dem
stellvertretenden Vorsitzenden unterschrieben wird. Das Wahlprotokoll wird durch
den Wahlleiter und mindestens zwei Wahlhelfer unterschrieben und dem Protokoll
beigefügt.

\item Der Landesparteitag wählt für die anstehende Amtsperiode des Vorstandes
mindestens zwei Rechnungsprüfer, die den finanziellen Teil des
Tätigkeitsberichtes des Vorstandes vor der Beschlussfassung über ihn prüfen. Das
Ergebnis der Prüfung wird dem Landesparteitag verkündet und zu Protokoll
genommen. Danach sind die Rechnungsprüfer aus ihrer Funktion entlassen.

\item Es können außerordentliche Parteitage statt finden. Die Einberufung
erfolgt aufgrund eines Vorstandsbeschlusses. Wenn ein Zehntel der Piraten,
mindestens aber zehn Piraten es beim Vorstand beantragen, muss dieser binnen 2
Wochen einen Parteitag einberufen. Dies geschieht schriftlich mit einer Frist
von zwei Wochen unter Angabe der Tagesordnung und des Tagungsortes.
\end{enumerate}

\subsubsection{§ 9c - Gebietsversammlung}
\begin{enumerate}
\item Eine Gebietsversammlung ist die Versammlung aller Piraten eines Landkreises, einer Gemeinde, einer Stadt, eines Ortsteils oder Stimm- bzw. Wahlkreises im Bundesland Sachsen-Anhalt.

\item Die Gebietsversammlung ist ein Organ der untersten existierenden
Gliederung, die das Gebiet vollständig umfasst. Diese Gliederung wird im
folgenden als "zuständige Gliederung" bezeichnet. Ist das Gebiet identisch mit
dem Gebiet der zuständigen Gliederung, so ist eine Mitgliederversammlung
stattdessen durchzuführen.

\item Der Vorstand der zuständigen Gliederung vertritt die Interessen der
Gebietsversammlung nach Maßgabe ihrer Beschlüsse, sofern die Gebietsversammlung
keine Personen aus ihrer Mitte damit beauftragt.

\item Die Gebietsversammlung entscheidet über
\begin{enumerate}
\item ausschließlich das Gebiet betreffende politische Fragen

\item gegebenenfalls weitere ihr nach der Satzung der zuständigen Gliederung
zukommende Aufgaben
\end{enumerate}

\item Stimmberechtigt ist jeder Pirat, dessen angegebener Wohnsitz im Gebiet der
Gebietsversammlung liegt. Die Bestimmungen in §4 (4) der Bundessatzung gelten
entsprechend.

\item Eine Gebietsversammlung wird vom Vorstand der zuständigen Gliederung
einberufen, wenn
\begin{enumerate}
\item der betreffende Vorstand es beschließt

\item mindestens 10%, aber mindestens drei Mitglieder, des Gebiets es verlangen
\end{enumerate}

\item Gibt sich die Gebietsversammlung keine eigene Wahl- und Geschäftsordnung,
gilt die aktuelle Wahl- und Geschäftsordnung der zuständigen Gliederung.

\item Eine Gebietsversammlung ist beschlussfähig, wenn mindestens 5\%, aber
mindestens drei Piraten, des Gebiets akkreditiert sind.

\item Für die Einladung zu einer Gebietsversammlung gelten die gleichen
Regelungen wie zur Mitgliederversammlung der zuständigen Gliederung. Die Satzung
der zuständigen Gliederung kann jedoch abweichende Regelungen beschließen.
\end{enumerate}

\subsubsection{§ 9d - Aufstellungsversammlung}
\begin{enumerate}
\item Die Aufstellungsversammlung ist die Versammlung zur Bewerberaufstellung
für die Wahlen zu Volksvertretungen.

\item Die Bewerberaufstellung für die Wahlen zu Volksvertretungen erfolgt nach
den Regularien der einschlägigen Gesetze sowie den Parteisatzungen der
Gliederungen, die den betreffenden Stimm- bzw. Wahlkreis vollständig umfassen.

\item Die Regelungen in § 9c mit Ausnahme von Absatz 4 und 5 gelten
entsprechend auch für Aufstellungsversammlungen.

\item Stimmberechtigt ist jedes Mitglied, dass zum Zeitpunkt der Wahl der
Volksvertretung wahlberechtigt ist.
\end{enumerate}

\subsection{§ 10 - Bewerberaufstellung für die Wahlen zu Volksvertretungen}
\begin{enumerate}
\item Die Bewerberaufstellung für die Wahlen zu Volksvertretungen wird durch die
Aufstellungsversammlung durchgeführt. Näheres regelt § 9d.
\end{enumerate}

\subsection{§ 11 - Satzungs- und Programmänderung}
\begin{enumerate}
\item Änderungen der Landessatzung und dem Grundsatzprogramm können nur von
einem Landesparteitag mit einer 2/3 Mehrheit beschlossen werden. Besteht das
dringende Erfordernis einer Satzungsänderung zwischen zwei Landesparteitagen, so
kann die Satzung auch geändert werden, wenn mindestens 2/3 der Piraten sich mit
dem Antrag/den Anträgen auf Änderung schriftlich (Brief, Email oder Fax)
einverstanden erklären.

\item Über einen Antrag auf Satzungs- oder Programmänderung auf einem
Landesparteitag kann nur abgestimmt werden, wenn er mindestens drei Wochen vor
Beginn des Landesparteitages beim Vorstand eingegangen ist.

\item Ausgenommen von dieser Frist sind Änderungsanträge, die sich auf nach
Punkt (2) beantragte Programmanträge beziehen. Diese können auch vor Ort
gestellt werden.

\item Vom Landesparteitag kann ein eigenes Grundsatzprogramm für den
Landesverband sowie Wahlprogramme für Kommunal und Landtagswahlen verabschiedet
werden. Diese dürfen dem Grundsatzprogramm der Piratenpartei Deutschland nicht
widersprechen.
\end{enumerate}

\subsection{§ 12 - Auflösung und Verschmelzung}
\begin{enumerate}
\item Die Auflösung oder Verschmelzung regelt die Bundessatzung. 
\end{enumerate}

\subsection{§ 13 - Parteiämter}
\begin{enumerate}
\item Die Regelung der Bundessatzung zu den Parteiämtern findet Anwendung. 
\end{enumerate}

\subsection{§ 14 - Verbindlichkeit dieser Landessatzung}
\begin{enumerate}
\item Die Satzungen der Untergliederungen des Landesverbandes müssen mit den
grundsätzlichen Regelungen dieser Satzung übereinstimmen.

\item Widerspricht ein Teil dieser Satzung dem Gesetz oder der Bundessatzung, so
bleiben die restlichen Bestimmungen trotzdem in Kraft.
\end{enumerate}


\section{Abschnitt B: Finanzordnung}
\begin{enumerate}
\item Es gilt im Wesentlichen die Bundesfinanzordnung.

\item Der Vorstand ist dem Vier-Augen-Prinzip verpflichtet. Jede Transaktion muß
von zwei Vorstandsmitgliedern unterzeichnet werden, wobei der übrige Vorstand
unverzüglich in Kenntnis zu setzen ist, oder durch einen Vorstandsbeschluss
gedeckt sein.

\item Der Schatzmeister des Landesverbandes kann gegen Transaktionen sein Veto
einlegen, wenn es die Finanzlage erfordert.

\item Der Schatzmeister des Landesverbandes kann von untergeordneten
Gliederungen alle für den Rechenschaftsbericht notwendigen Daten einfordern.
Sollte dies nicht möglich sein, hat er zeitnah Ordnungsmaßnahmen zu beantragen.
\end{enumerate}

\subsection{§1 - Umlage Parteienfinanzierung }

Die Gelder aus der Parteienfinanzierung werden auf Landesebene nach folgendem
Schlüssel umgelegt:
\begin{enumerate}
\item 10\% der Parteienfinanzierung verbleibt bis zur nächsten Abschlagszahlung,
mindestens jedoch für ein Jahr, als Rücklage beim Landesverband. Aufgelöste
Rücklagen werden zur aktuellen Abschlagszahlung addiert und entsprechend diesem
Schüssel umgelegt.

\item Vom verbleibenden Betrag gehen 50\%, mindestens jedoch ein Sockelbetrag
von 3600 EUR per anno, an den Landesverband. Der Restbetrag geht an die
untergliederten Kreisverbände.

\item Die Verteilung des Anteils der Kreisverbände erfolgt zu je einem Drittel
nach Sockel, nach Einwohner und nach Fläche der Kreisverbände.
\begin{enumerate}
\item Der Sockelanteil eines Kreisverbandes berechnet sich aus dem Verhältnis
Anzahl der politischen Kreise des Kreisverbandes zu Anzahl der politischen
Kreise des Landes.

\item Der Anteil nach Einwohner berechnet sich aus dem Verhältnis Einwohnerzahl
des Gebietes des Kreisverbandes zu Einwohnerzahl des Landes.

\item Der Anteil nach Fläche berechnet sich aus dem Verhältnis Fläche des
Gebietes des Kreisverbandes zu Fläche des Landes.
\end{enumerate}

\item Sofern in einem politischen Kreis noch kein Kreisverband existiert, wird
der entsprechende Betrag gegen ein virtuelles Unterkonto des Landesverbandes
gebucht. Von diesem Unterkonto sollen primär Aktionen in dem jeweiligen Gebiet
finanziert werden. Der Landesvorstand ist berechtigt diesen Betrag begründet
anderweitig zu verwenden.

\item Anspruch auf Auszahlung aus der Parteienfinanzierung besteht ab dem Monat
der Gründung eines Kreisverbandes.
\end{enumerate}

\subsection{§ 2 - Finanzrat }
\begin{enumerate}
\item Der Landesparteitag wählt einmal jährlich zwei Piraten des Landesverbandes
in den Finanzrat der Piratenpartei Deutschland.
\end{enumerate}


\section{Abschnitt C: Schiedsgerichtsordnung}
\begin{enumerate}
\item Für das Landesschiedsgericht gilt die Schiedsgerichtsordnung der
Bundespartei.
\end{enumerate}


\section{Abschnitt D: Liquid Democracy}
\begin{enumerate}
\item Die Piratenpartei Deutschland Sachsen-Anhalt nutzt zur Willensbildung über
das Internet eine geeignete Software. Diese muss die "Anforderungen für den
Liquid Democracy Systembetrieb" erfüllen, welche vom Vorstand beschlossen
werden.
Die Mindestanforderungen sind:
\begin{enumerate}
\item Jedes Mitglied muss die Möglichkeit haben, Anträge im System zu stellen.
Zulassungsquoren und Antragskontingente sind zulässig, müssen jedoch für alle
Mitglieder gleich sein.
\item Das System muss ohne Moderatoren auskommen.
\item In das System eingebrachte Anträge dürfen nicht gegen den Willen des
Antragsstellers von anderen Mitgliedern verändert oder gelöscht werden können.
\item Jedem Mitglied muss es innerhalb eines bestimmten Zeitraums möglich sein,
Alternativanträge einzubringen.
\item Das eingesetzte Abstimmungsverfahren darf Anträge, zu denen es ähnliche
Alternativanträge gibt, nicht prinzipbedingt bevorzugen oder benachteiligen.
Mitgliedern muss es möglich sein, mehreren konkurrierenden Anträgen gleichzeitig
zuzustimmen. Der Einsatz eines Präferenzwahlverfahrens ist hierbei zulässig.
\item Es muss möglich sein, die eigene Stimme mindestens themenbereichsbezogen
durch Delegation an ein anderes Mitglied zu übertragen. Diese Delegationen
müssen jederzeit widerrufbar sein und übertragenes Stimmgewicht muss weiter
übertragen werden können. Selbstgenutztes Stimmgewicht darf nicht weiter
übertragen werden.
\end{enumerate}

\item Der Vorstand stellt den dauerhaften und ordnungsgemäßen Betrieb des
Systems sicher.

\item Jedem Mitglied ist Einsicht in den abstimmungsrelevanten Datenbestand des
Systems zu gewähren. Während einer Abstimmung darf der Zugriff auf die
jeweiligen Abstimmdaten anderer Mitglieder vorübergehend gesperrt werden.

\item Die Organe sind gehalten, das Liquid Democracy System zur Einholung von
Empfehlungen zur Grundlage ihrer Beschlüsse zu nutzen und vom diesen
Empfehlungen abweichende Entscheidungen zu begründen. Das Schiedsgericht ist
davon ausgenommen.

\item Die Organe der Partei sind angehalten, die Anträge, die im Liquid
Democracy System positiv beschieden wurden, vorrangig zu behandeln.

\item Teilnahmeberechtigt ist jeder Pirat, der nach der Satzung Mitglied der
Piratenpartei Sachsen-Anhalt ist. Jeder Pirat erhält genau einen persönlichen
Zugang, der nur von ihm genutzt werden darf.

\item Verstößt ein Nutzer wiederholt und in erheblichem Maße gegen die
Nutzungsbedingungen des Systems, so kann der Vorstand als Ordnungsmaßnahme dem
Nutzer auf Zeit das Recht entziehen, Anträge oder andere Texte in das System
einzustellen. Im Falle technischer Angriffe auf das System, die von einem
angemeldeten Benutzer ausgehen, kann dieses Benutzerkonto durch Administratoren
vor+bergehend gesperrt werden.
\end{enumerate}